\chapter{História e Desenvolvimento}\label{cha:history-development}


\section{História do Produto}

O XCSoar iniciou como um produto comercial desenvolvido por Mike Roberts (UK) e conseguiu uma boa fatia de mercado durantes anos e com várias versões, a última sendo a {\bf Versão~2}.
Por razões pessoais impediram-no de continuar sendo capaz de dar assistência ao produto e em 2004 anunciou o licenciamento do código-fonte para a licença pública GNU, com o XCSoar {\bf Versão~3}.  Um site de apoio no Yahoo Groups foi montado e o projeto de fonte aberta começou a ganhar interesse e introduzir desenvolvedores.

Em março de 2005, o programa foi substancialmente melhorado e o resultado foi o lançamento da {\bf Versão~4.0}.  Neste momento, a coordenação do desenvolvimento do código fonte se tornou difícil e consumindo muito tempo.  Decidiu-se para mover o projeto para o SourceForge, onde todos os softwares podem ser gerenciados por um sistema de gerenciamento de versões concorrentes.

Em julho de 2005, a {\bf Versão~4.2} foi lançada e apresentava alguns problemas de compatibilidade com alguns PDAs e configurações de hardware de GPS.

Em setembro de 2005, a {\bf Versão~4.5} foi lançada.  Continha grandes melhorias para a interface do usuário, incluindo a introdução de ‘entrada de eventos’ e arquivos de traduções de idiomas.

Em abril de 2006, a {\bf Versão~4.7} foi lançada para os usuários do Altair.  Tinha melhoria na estabilidade e desempenho, bem como vários problemas consertados e um novo método para lidar com as janelas, baseado nos arquivos XML.

Em setembro de 2006, a {\bf Versão~5.0} foi lançada para todas as plataformas, Altair, PC, PDA.  Esta versão continha muitas melhorias e novas funções e era baseada no teste exaustivo em vôo e simulação.

Em setembro de 2007, a {\bf Versão~5.1.2} foi lançada para todas as plataformas, Altair, PC, PDA.  Esta versão continha muitas melhorias e novas funções e era baseada no teste exaustivo em vôo e simulação.  As melhorias principais incluem um novo formato de arquivo de mapa, incorporando a compressão JPG2000, apoio online para competições, dispositivos adicionais suportados, tela de radar FLARM e melhor estabilidade, confiabilidade e precisão de cálculos de prova.  Muitas necessidades dos usuários foram incorporadas nesta versão.

Em fevereiro de 2008, a {\bf Version~5.1.6} foi lançada.  Continha inúmeros reparos em bus e melhoria na interface do usuário, expandindo as funcionalidades RASP e AAT. 

Em março de 2009, a {\bf Versão~5.2.2} foi lançada.  Melhoria na interface do usuário e foram introduzidas muitas funcionalidades: os arquivos IGC foram digitalmente assinados para validação em provas online como OLC.  Pela primeira vez, disponível para PNA com Windows-CE.  O FLARM foi integrado à exibição do mapa com suporte para a rede FLARM e banco de dados de identificação.  Os desenvolvedores podem agora facilmente compilar o XCSoar através do Linux em desktops. 

Em agosto de 2009 a {\bf Versão~5.2.4} foi lançada com ajustes internos e melhorias.

Em dezembro de 2010, a {\bf Versão~6.0} foi lançada.  Seguindo uma extensiva reescrita de quase todo o programa, fazendo muitas melhorias na estabilidade e desempenho sendo o tempo de início do programa dramaticamente reduzido.  Muitas grandes funcionalidades foram adicionadas, inclusive uma vasta ferramenta de edição e prova, suporte AAT, novas exibições de FLARM e assistente de termal.   Novos idiomas foram introduzidos e novas traduções podem agora geradas pelos usuários.

Significativamente, a reescrita permitiu o XCSoar rodar em sistemas Unix e dispositivos Android e o uso de ferramentas modernas de compilação melhorou o desempenho do programa na geração atual de dispositivos.

Em março de 2011, a {\bf Versão~6.0.7b} foi lançada, sendo a primeira versão oficialmente para Android.

Início de 2013: com os primeiros resultados da {\bf Versão~6.4.5.} foram publicadas as traduções em alemão e francês do manual na página do XCSoar.

Fevereiro de 2012: o XCSoar é campeão mundial na Argentina!  Tobias deu uma entrevista com Santi que pode ser lida no link abaixo:
\xcsoarwebsite{/discover/2013/02/05/WGC_Argentina.html}.

Abril de 2013: a {\bf Versão~6.6} foi lançada.  Seis anos após iniciar o retrabalho do mecanismo da versão 6  do XCSoar, o resultado se tornou visível no nível do usuário.  Um marco notável no uso e no valor do XCSoar, incluindo uma visualização de corte transversal do mapa.

Maio de 2013: a instalação do XCSoar para Android conta com o lançamento estável passando da marca de 30.000 downloads.

Março de 2015: a {\bf Versão~6.8} foi lançada.  Incluia também a tradução para o português do Brasil.


\section{Seja envolvido}

O sucesso do projeto é o resultado de muitas maneiras de contribuição. Você não precisa ser um desenvolvedor de softwares para ajudar.

Em geral, há cinco formas principais de contribuição, ao invés de trabalhar sozinho no software:

\begin{description}
\item[Dê retorno]
ideias, informações sobre erros e críticas construtivas são muito bem vindas e muito úteis.
\item[Sugestões de ajustes]
como o XCSoar é muito configurável, nós confiamos em alguns usuários para informar como gostariam que os ajustes do programa fossem feitos.  A seleção das infoboxes, layouts, botões e funções necessitam de algum design e disponibilizando isto para os desenvolvedores e outros usuários irá nos ajudar a fornecer um bom ajuste padrão.
\item[Integridade de dados]
os arquivos de espaço aéreo e waypoints precisam ser atualizados e freqüentemente necessitam de pessoas com conhecimento para fazer isso.
\item[Promoção]  quanto mais usuários o software tiver, melhor o produto será.  Quanto mais pessoas usarem o software e fornecerem retorno, as falhas são achadas mais facilmente e as melhorias ocorrem em tempo menor.  Você pode ajudar aqui, por exemplo, mostrando o software aos outros e promovendo demonstrações e treinamentos em seu clube.
\item[Documentação]  geralmente este manual está sempre desatualizado e necessitamos de ajuda para mantê-lo atualizado.
\end{description}


\section{Filosofia de código aberto}

Há vários benefícios em ter um software de código aberto como o XCSoar:

\begin{itemize}
\item Primeiramente, é livre e os pilotos podem experimentar o software sem custo e decidir se é apropriado para suas necessidades.  Os pilotos são livres para copiar o programa para qualquer Pocket PC, PC ou EFIS sem qualquer custo.
\item Você tem acesso ao código-fonte, portanto é livre para alterar o software ou usar partes do mesmo em novos programas.
\item Tendo disponível o código fonte na internet significa que está sujeito a uma grande análise e portanto, falhas são facilmente e rapidamente reparadas.
\item Um grande grupo de desenvolvedores disponíveis a ajudar na resolução de problemas e rapidamente implementar novas características assim que requisitados.
\item Software de código aberto sob a licença pública GNU não pode a qualquer momento se tornar fonte fechada.  Portanto, usando este software você não terá custos de software no futuro.
\end{itemize}

Os termos completos do acordo de licença para o XCSoar são fornecidos no Anexo ~\ref{cha:gnu-general-public}.

O desenvolvimento do XCSoar desde que a versão de fonte aberta foi lançada é inteiramente voluntário.  Não pode excluir desenvolvedores individuais ou organizações que ofereçam serviços comerciais de fornecer suporte.  O espírito do projeto entretanto sugere que nestes casos, que os serviços comerciais devem ser direcionados a produzir algum benefício de volta à grande comunidade de usuários.


\section{Processo de desenvolvimento}

Nós tentamos incorporar novas funções o mais rápido possível.  Isto tem que ser balanceado pelas necessidades de não mudar substancialmente a interface sem avisos apropriados, de forma que os usuários que façam a atualização não fiquem chocados.  Isto significa que, quando introduzimos um botão novo na versão 4.5, foi necessário distribuir também um arquivo que permitia aos usuários terem o botão atribuído à sua função de origem.

O XCSoar, sendo usado em vôo como um software especial, pode ser considerado como um sistema de missões críticas em tempo real.  Isto tem uma alta relevância para os desenvolvedores, para que desempenhem vários testes antes de lançar as mudanças ao público.

O teste em vôo é certamente a melhor forma de testar, mas nós somos aptos a conduzir uma batelada de testes usando o XCSoar no carro e mais recentemente, repetindo os registros de vôo IGC.

Em geral, não queremos nunca que o programa trave ou desligue, e se acontecer durante o teste, então o que causou a falha tem que ser solucionado como sendo da mais alta prioridade.

Todos os desenvolvedores do software estão em contato permanentemente uns com os outros através da lista de desenvolvedores SourceForge no e-mail abaixo:

\begin{quote}
\url{xcsoar-devel@lists.sourceforge.net}
\end{quote}
Nós tentamos coordenar nossas atividades para prevenir conflitos e esforços duplicados, trabalhando juntos como um time.  Se você pode se envolver no desenvolvimento do software, mande um e-mail para os desenvolvedores.


\section{Base do usuário}

Quem está usando o XCSoar?  Boa pergunta e difícil de responder.  Sendo que não há pagamento pelo produto – a maioria das pessoas baixam o programa anonimamente – é difícil para qualquer um rastrear quantas pessoas estão usando o XCSoar.  

As estatísticas do site principal indicam que tem uma média de aproximadamente 20 downloads por dia entre junho de 2005 e junho de 2006 e oitenta downloads entre junho de 2006 e setembro de 2007.  Olhando para quantas pessoas baixaram os pacotes de terreno e topologia do site, indicam que é usado em muitos países e em todos os continentes.

O XCSoar é usado por vários tipos de pilotos, incluindo pilotos iniciantes e até pilotos de competição.  Há vários pilotos de computador que usam o XCSoar com muitos simuladores de vôo, como o Condor.


