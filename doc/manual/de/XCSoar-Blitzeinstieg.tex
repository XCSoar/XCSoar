\documentclass[german,a4paper,12pt]{refrep}
\settextfraction{1}
\usepackage[utf8]{luainputenc}
\usepackage{xcolor}
\usepackage{booktabs}
\usepackage{longtable}
\usepackage{tabularx}
\usepackage{rotating}
\usepackage{multicol}
\usepackage{multirow}
\usepackage[disable]{todonotes}
\usepackage{tocloft}
\usepackage{hyperref}
\usepackage{gensymb}
\makeatletter

% Font selection
\renewcommand{\familydefault}{\sfdefault}
\fontfamily{phv}\selectfont

% Reference colors
\hypersetup{
  unicode=true,
  linkcolor=blue, % internal links
  linktocpage=true, % only page numbers
  urlcolor=blue, % external links
  citecolor=green,
  filecolor=magenta,
  bookmarks=true,
  bookmarksnumbered=true,
  breaklinks=true, % wrap links is Ok
  colorlinks=true,
  pdftoolbar=true,
  pdfmenubar=true,
  pdfnewwindow=true
}

% Some shortcuts
\newcommand\xc{\textsf{XCSoar}}
\newcommand\fl{\textsf{Flarm}}
\newcommand\al{\textsf{Altair}}

% Define command to insert XCSoar website
\newcommand{\xcsoarwebsite}[1]{\url{https://xcsoar.org#1}}
\newcommand{\xcsoarforum}[1]{\url{https://forum.xcsoar.org#1}}

% Define command to insert tip image
\newcommand{\tip}[0]{\marginlabel{\parbox{1.1cm}{\includegraphics[width=0.7cm]{figures/reminder.pdf}}}}

% Define command to insert gesture image
\newcommand{\gesture}[1]{\marginlabel{{\it#1
}\parbox{1.3cm}{\includegraphics[width=0.7cm]{figures/gesture.pdf}}}}

% Define command to insert specific gesture image
\newcommand{\gesturespec}[1]{\marginlabel{\parbox{1.3cm}{\includegraphics[width=0.9cm]{figures/#1.png}}}}

% Define command to insert warning image
\newcommand{\warning}[0]{\marginlabel{\parbox{1.3cm}{\includegraphics[width=0.9cm]{figures/warning.pdf}}}}

% Define command to insert Achtung image
\newcommand{\achtung}[0]{\marginlabel{\parbox{1.3cm}{\includegraphics[width=2.5em]{figures/warning.pdf}}}}

% Define command to insert a flash image
\newcommand{\blitz}[0]{\marginlabel{\parbox{1.3cm}{\includegraphics[height=2.0em]{figures/reminder.pdf}}}}

% Define command to insert a stop
\newcommand{\halt}[0]{\marginlabel{\parbox{1.3cm}{\includegraphics[height=2.0em]{figures/warning.pdf}}}}

% Define command to reference a configuration item
\newcommand{\config}[1]{\marginlabel{\ref{conf:#1}
\parbox{1.3cm}{\includegraphics[width=0.8cm]{figures/config.pdf}}}}

% Define command to draw a sketch on the margin
\newcommand{\sketch}[1]{\marginpar{\parbox{4.0cm}{\includegraphics[angle=0,width=1.0\linewidth,keepaspectratio='true']{#1}}}}
\newcommand{\smallsketch}[1]{\marginpar{\includegraphics[angle=0,keepaspectratio='true']{#1}}}


% Potentially overdue ``InfoBox'' style macro
\newcommand{\InfoBox}[0]{{InfoBox}}

% Enumerated todo's for the todonotes package
\newcounter{todocounter}
\newcommand{\todonum}[2][]{\stepcounter{todocounter}\todo[#1]{\thetodocounter: #2}}

\maxipagerulefalse

% Include XCSoar header and footer settings
\usepackage{calc}

% Colors
\definecolor{buttongray}{rgb}{0.831,0.816,0.784}


% A set of boxes, buttons etc.
%
% Simple gray button
\newcommand{\blink}[0]{$\triangleright$}

\newcommand{\bmenug}[1]{
  \fcolorbox {black}{buttongray}{{\footnotesize\textsf{#1}}}
}

\newcommand{\button}{\bmenug}
\newcommand{\bmenu}{\bmenug}

\newcommand{\bmenuw}[1]{
  \fcolorbox {black}{white}{{\footnotesize\textsf{#1}}}
}

\newcounter{mboxwidth}
\setcounter{mboxwidth}{14}
\newcommand{\setbuttonwidth}[1]{\setcounter{mboxwidth}{#1}}

\newcommand{\bmenut}[2]{    % Menu button (two lined)
  \fcolorbox {black}{buttongray}{
    \makebox[\value{mboxwidth}mm][c]{
      \begin{tabular}{c}
        {\footnotesize\textsf{#1}}\\
        {\footnotesize\textsf{#2}}
      \end{tabular}
    }
  }
}

\newcommand{\bmenuth}[3]{    % Menu button (three lined)
  \fcolorbox {black}{buttongray}{
    \makebox[\value{mboxwidth}mm][c]{
      \begin{tabular}{c}
        {\footnotesize\textsf{#1}}\\
        {\footnotesize\textsf{#2}}\\
        {\footnotesize\textsf{#3}}
      \end{tabular}
    }
  }
}

\newcommand{\bmenus}[1]{    % Menu button (single line)
  \fcolorbox {black}{buttongray}{
    \makebox[\value{mboxwidth}mm][c]{
      \begin{tabular}{c}
        {\footnotesize\textsf{#1}}\\
        \\
      \end{tabular}
    }
  }
}

\newcommand{\infobox}[1]{    % Normal Info box in text
  \fcolorbox {black}{white}{\makebox[1.7cm][c]{\textsf\strut #1}}
}

% Some more convenience
\newenvironment{jspecs}{ % Description spacing
  \itemsep=2pt\topsep=3pt\partopsep=3pt\parskip=0pt
  \begin{description}
    \itemsep=2pt\topsep=3pt\partopsep=3pt\parskip=0pt
}{\end{description}}

\newcommand{\jindent}[2]{ % Extra list spacing
  \noindent\makebox[0pt][r]{{#1}\hspace*{\marginparsep}}
  \parbox[t]{0.95\linewidth}{#2}\par
}

\widowpenalty=1000
\clubpenalty=1000

% the command \version prints the XCSoar version number
\newcommand{\version}{\begingroup\catcode`\_=\active\input{VERSION.txt}\endgroup}

% Define command to put a menu label on the margin
% aligned left
\newcommand{\menulabel}[1]{\marginpar{\parbox{5.0cm}{\raggedright #1}}}
% aligned right
\newcommand{\menulabelr}[1]{\marginpar{\parbox{4.05cm}{\raggedleft #1}}}

% Define some colors
\definecolor{AirspaceYellow}{rgb}{.99,.99,.19}
\definecolor{AirspaceRed}{rgb}{.99,.19,.19}

\usepackage[german]{babel}
\usepackage{wasysym}
\usepackage{graphicx}                      % Anleitung hierzu  in c:\Program Files\MiKTeX 2.7\doc\latex\graphics\grfguide.pdf
\usepackage{paralist}
% Wir wollen einen Index:
\usepackage{makeidx}\makeindex
% Wir wollen aktive Links und einige Dokumentinformationen:
\hypersetup{
    pdftitle={XCSoar Blitzeinstieg},
    pdfauthor={OH},
    pdfsubject={Subject},
    pdfproducer={OH},
    pdfkeywords={XCSoar}{Schnelleinstieg}{Segelflug}{Endanflug},
}
\usepackage[right]{eurosym}% Eurosymbol, rechts neben Text mit \EUR{427,45}, symbol alleine mit\eur{}
\parindent0mm%kein Einzug-häßlich
\setlength{\baselineskip}{1em}

% Set the page title
\usepackage{calc}
\usepackage{fancyhdr}

\newcommand{\xcsoarheader}[1]{
  \pagestyle{fancy}

  % Add XCSoar User Manual title to the header
  \fancyhead[L]{\hspace*{-\marginparsep}\hspace*{-\marginparwidth}\em#1}

  % Add page number to the footer (centered)
  \fancyfoot{}
  \fancyfoot[R]{\thepage}
}

% No line between content and header
\renewcommand{\headrulewidth}{0pt}

\fancypagestyle{plain}{
  % Clear fancy header and footer
  \fancyhf{}

  % Add page number to the footer (centered)
  \fancyfoot[R]{\thepage}

  % No line between content and header
  \renewcommand{\headrulewidth}{0pt}
}

\xcsoarheader{XCSoar Blitzeinstieg}

%
\fboxrule0.4mm                                                      % Breite des Fbox-Rahmens
%
%##############  Makros und  Menus ..##############
%
\newcommand{\bc}{\begin{center}}
\newcommand{\ec}{\end{center}}
%
\newcommand{\bmg}{\bmenug}
\newcommand{\bmw}{\bmenuw}
\newcommand{\bmt}{\bmenut}
\newcommand{\bms}{\bmenus}
\setcounter{mboxwidth}{16} % Etwas groesser als standard
%
%##############  Erweiterte Makros und  Menus ..##############
\newcommand{\nav}[3]{\bmt{Nav}{#1/2}{\LARGE$\triangleright$}~\bmt{#2}{#3}}%NavMenu
\newcommand{\ansi}[3]{\bmt{Ansicht}{#1/2}{\LARGE$\triangleright$}~\bmt{#2}{#3}}%AnsichtMenu
\newcommand{\konf}[3]{\bmt{Konfig}{#1/3}{\LARGE$\triangleright$}~\bmt{#2}{#3}}%KonfigNavMenu
\newcommand{\info}[3]{\bmt{Info}{#1/3}{\LARGE$\triangleright$}~\bmt{#2}{#3}}%InfoMenu

\newcommand{\sk}[0]{%  Häufig benutzt .. konf-konf-system-einstellungen
    \fcolorbox {black}{buttongray}{
    \makebox[1.6cm][c]{
        \begin{tabular}{c}
        {\footnotesize\textsf{Konfig}}\\
        {\footnotesize\textsf{2/3}}
        \end{tabular} } }
    {\LARGE$\triangleright$}\hspace{0.0075em}
    \fcolorbox {black}{buttongray}{
    \makebox[1.6cm][c]{
        \begin{tabular}{c}
        {\footnotesize\textsf{System}}\\
        {\footnotesize\textsf{Einstellung}}
        \end{tabular} } }
}
%
%##############  Titelseite des Blitzes .##############
%
\hyphenation{Kon-fi-gu-ra-tions-files Onkel-hotte}
\setcounter{secnumdepth}{1}
\renewcommand*{\thesection}{\textsf{\arabic{section}}}
\begin{document}
\thispagestyle{empty}
\vspace*{3em}
\begin{center}
{\Huge Die Blitzanleitung zum Glücklichsein mit}

\vspace{3em}
%\includegraphics[angle=0,width=0.9\linewidth,keepaspectratio='true']{figures/xcsoar-title.png}
% xcsoar-title.png: 964x394 pixel, 100dpi, 24.49x10.01 cm, bb=0 0 694 284
\includegraphics[angle=0,width=0.5\linewidth,keepaspectratio='true']{figures/blitzlogo.png}

\vspace{2em}
{\Huge \xc,\\[0,5em]  \textbf{dem} Segelflugrechner}
\end{center}

\newpage

%%%%%--------------------------------------------------------------------%%%%%
%%%%%--------------------------------------------------------------------%%%%%
%%%%%--------------------------------------------------------------------%%%%%
%%%%%--------------------------------------------------------------------%%%%%
%%%%%--------------------------------------------------------------------%%%%%
\section{{\textsf XCSoar} Einrichten}\label{ch:XCSeinrichten}
%%%%%--------------------------------------------------------------------%%%%%
Nach dem Herunterladen der entsprechenden Programmversion {\textsf (Android, PC, Linux, Mac)}
sollte in ein entsprechendes Verzeichnis entpackt bzw. installiert werden.\\
Je nach Betriebssystem sollte ein Verzeichnis gewählt werden, welches vollen Zugriff hat.
Es ist zwingend ein Unterverzeichnis {\verb XCSoarData} zu erstellen, in dem sämtliche
Konfigurationsfiles,  Luftraumfiles, Wegpunktfiles und das XCM-Terrain/Geländefile
untergebracht wird.\\

Die Installationsroutinen von Android, Windows Mobile, PPC, PC machen dies von alleine. Der
Betrieb von einer SD-Karte ist möglich. Es können auch andere über- oder Unterverzeichnisse
gewählt werden, für den reibungslosen Betrieb und oder/ Updates empfiehlt es sich jedoch bei
den Vorgaben zu bleiben.\\


\begin{compactitem}
\item[1.] Herunterziehen von der Homepage:

 {\large\url{http://www.xcsoar.org}}
\item[2.] XCM File laden (auch auf der Homepage)
\item[3.] Wegpunkte laden (cup format oder openairformat)
\item[4.] Luftraumfile laden:

 {\large\url{http://www.daec.de/fachbereiche/luftraum-flugbetrieb/luftraumdaten}}
\item[5.] Folgende Kombination eingeben
\end{compactitem}

\bc\sk\blink\bmw{Standortdatei}\blink\bmw{Standortdatei}\ec\index{Kartendatei}\index{Luftraumdateie}\index{Wegpunktdatei}

Hier dann die entsprechenden Files auswählen. Falls keine Files zur Auswahl bereitstehen, den
Pfad überprüfen. Alles Files müssen zwingend in {\large\texttt{/XCSoarData}} stehen!!! Mehr dazu im
Handbuch.

\textbf{Wichtig:} in Kap.~\ref{ch:generell} stehen noch ein paar \textbf{wirklich} lesenswerte Anmerkungen!!!

\subsection{\textcolor{blue}{Dateimanager}}\index{Dateimanager}
\textbf{Achtung!}

Der Dateimanager funktioniert unter Android erst ab Android 2.2!!!


Mit dem in \xc~integrierten Dateimanager ist es möglich --eine funktionierende Internetverbindung vorausgesetzt--
fehlende Dateien direkt und einfach herunterzuladen:

\bc\konf{2}{Datei}{manager}\blink\bmw{Hinzufügen}\ec
Hier öffnet sich eine Liste mit etlichen Karten, Lufträumen und Wegpunktfiles, die  Du unverzüglich
downloaden und in \xc~an die richtige Stelle kopieren kannst. 
Mit \bmw{Download} werden die Dateien dann heruntergeladen und installiert. 

Sogar eine Karte mit häufigen Thermik-auslösern und -Gegenden sind dabei :

({\large\texttt{\textcolor{blue}{Thermal\_Spaces\_EU.txt}}} und {\large\texttt{\textcolor{blue}{OpenAIP-Hotspots.cup}}} )  


\subsection*{\textcolor{blue}{Sprache, Schriften, Klänge  einstellen}}
Normalerweise lädt \xc~ die Sprache des Betriebsystemes, auf dem es installiert ist.
Wenn Ihr dennoch auf Norwegisch fliegen wollt:

\bc\sk\blink\bmw{Aussehen}\blink\bmw{Sprache,Eingabe}\ec

Hier könnt Ihr nach Herzenslust Sprachen und Schriften einstellen, auch die Möglichkeit, eine
eigenen Datei anzugeben, welche bevorzugte Klänge zu bestimmten Situationen abspielt,
besteht hier.
%%%%%--------------------------------------------------------------------%%%%%
%%%%%--------------------------------------------------------------------%%%%%
%%%%%--------------------------------------------------------------------%%%%%
%%%%%--------------------------------------------------------------------%%%%%
%%%%%--------------------------------------------------------------------%%%%%
\newpage\section{Flugzeug}
%%%%%--------------------------------------------------------------------%%%%%
\subsection*{\textcolor{blue}{Polare einrichten}}
\xc~ besitzt intern bereits eine Liste mit Polaren. Die Auswahl erfolgt mit:
\bc\sk\blink\bmw{Einstellung}\blink\bmw{Polare}\ec
\begin{small}Der Index der Flugzeuge basiert auf der DAEC-Liste von April  {\textsf 2012} \end{small}
\subsection*{\textcolor{blue}{Wasserballast, Mücken, QNH, Temperatur eingeben}}
\bc\konf{1}{Flug}{Einstellung}\blink\bmw{Ballast/Mücken/QNH/Temperatur}\ec
\subsection*{\textcolor{blue}{Wind, Richtung, Stärke Berechnung, Drift}}
\bc\konf{1}{Wind}{Einstellung}\blink\bmw{Manuell/Richtung/Geschw\dots}\ec
\subsection*{\textcolor{blue}{Flugzeugname, Pilotenname}}
\bc\sk\blink\bmw{Einstellung}\blink\bmw{Logger Info}\ec
\subsection*{\textcolor{blue}{Logger, Flarm, Cambridge etc\dots NMEA einrichten}}
\bc\sk\blink\bmw{Einstellung}\blink\bmw{NMEA-Anschluss}\ec
\subsection*{\textcolor{blue}{Systemzeit einstellen}}
\bc\sk\blink\bmw{Einstellung}\blink\bmw{Zeit}\ec
%%%%%--------------------------------------------------------------------%%%%%
%%%%%--------------------------------------------------------------------%%%%%
%%%%%--------------------------------------------------------------------%%%%%
%%%%%--------------------------------------------------------------------%%%%%
%%%%%--------------------------------------------------------------------%%%%%
\newpage\section{Wegpunkte}
%%%%%--------------------------------------------------------------------%%%%%
Voraussetzung, dass mit Wegpunkten gearbeitet werden kann, ist, dass Ihr mindestens eine Wegpunktdatenbank installiert habt,
s. Kap.~\ref{ch:XCSeinrichten}
\subsection*{\textcolor{blue}{Wegpunkt/Flugplatz wählen}}\index{Heimatflugplatz}
\begin{itemize}
\item Einfacher Klick auf Bildschirm auf eine Stelle in der Nähe des gewünschten Platzes.
\item Es erscheint das Fenster ''Kartenelemente an diesem Ort''
\item Flugplatz aus der Liste auswählen,
\item \button{Ziele auf} drücken
\end{itemize}
\subsection*{\textcolor{blue}{Heimatflugplatz einstellen}}\index{Heimatflugplatz}
\begin{itemize}
\item Einfacher Klick auf Bildschirm auf eine Stelle in der Nähe des gewünschten Platzes.
\item Es erscheint das Fenster ''Kartenelemente an diesem Ort''
\item Flugplatz aus der Liste auswählen,
\item \button{Details} drücken
\item Eine Seite nach rechts mit dem schwarzen Pfeil: \quad${\Huge\RHD}$\quad und anschließend
\item \button{Setze als neuer Heimatort}
\end{itemize}
%%%%%--------------------------------------------------------------------%%%%%
%%%%%--------------------------------------------------------------------%%%%%
%%%%%--------------------------------------------------------------------%%%%%
%%%%%--------------------------------------------------------------------%%%%%
%%%%%--------------------------------------------------------------------%%%%%
\newpage\section{Flug}
%%%%%--------------------------------------------------------------------%%%%%
\subsection*{\textcolor{blue}{Separater Maßstab InfoBoxSeiten erst ab V 6.6}}\index{Separater Kartenmaßstab}
\bc\sk\blink\bmw{Kartenanzeige}\blink\bmw{Ausrichtung}\blink\bmw{Distinct Page Zoom}\ec

Es kann für jede Karten-Seite ein eigener Maßstab eingerichtet werden. Somit ist es möglich, z.B. für den Endanflug den Maßstab auf
96km zu setzen, während beim Kurbeln etwas mehr Details in einem kleineren Maßstab z.B.\  64km benutzt wird.

Hierzu muss der Schalter \textit{Distinct Page Zoom} auf ''ein'' gesetzt werden, s. oben.
%%%%%--------------------------------------------------------------------%%%%%
%%%%%--------------------------------------------------------------------%%%%%
%%%%%--------------------------------------------------------------------%%%%%
%%%%%--------------------------------------------------------------------%%%%%
%%%%%--------------------------------------------------------------------%%%%%
\newpage\section{Aufgaben / Wettbewerb}
%%%%%--------------------------------------------------------------------%%%%%
Hier können Voreinstellungen für die jeweiligen Aufgaben gemacht werden. Diese erscheinen
jedesmal als Default bei Erstellung einer neuen Aufgabe.

\bc\sk\blink\bmw{Aufgaben Voreinstellung}\blink\bmw{Regeln, Wendepunktypen}\ec

\subsection*{\textcolor{blue}{Aufgabe erstellen /Speichern / Laden / Anmelden /Fliegen}}\pdfbookmark[0]{Aufgaben erstellen}{Aufgaben}

\bc\nav{1}{Aufgabe}{}\blink\bmg{Verwalten}\blink\bmg{Neue Aufgabe}\ec
\begin{itemize}
\item[1.]Im gleichen Menü findest Du die Buttons für Laden (Durchsuchen), Speichern
und Anmelden (z.B.\ an Flarm, Logger  o.ä.) Nach dem Erzeugen einer neuen Aufgabe klappt zuerst das Fenster \bmg{Regeln} hoch.
Hier alles entsprechend anpassen. Wichtig! \bmw{Aufgabentyp} beachten!!
\item[2.] Mit \bmg{Wendepunkt}\blink\bmw{Wendepunkt hinzufügen} und dem
Wendepunktfilter die entsprechenden Punkte eingeben.
 \item[3.] Mit \bmg{Punkt bearbeiten} kann jederzeit der Typ und der Punkt selber
 bearbeitet werden.
\end{itemize}
\textbf{Wichtig:}

Mit \bmw{Schließen} verlässt Du die Aufgabe, welche dann aber {\textsf noch nicht} gespeichert ist!!

Besser daher: \bmw{Verwalten} und anschließend \bmw{Speichern}, dann hast Du die Aufgabe drin
und kannst letztlich \bmw{Fliegen}


\subsection*{\textcolor{blue}{Aufgaben  Startzeit und -fenster eingeben}}\pdfbookmark[0]{Startfenster}{Startfenster}
\bc\nav{1}{Aufgabe}{}\blink\bmw{Regeln\strut}\blink\bmw{Abflugzeit} etc\dots\ec
\subsection*{\textcolor{blue}{Aufgabe starten}}\index{Aufgabe starten}\pdfbookmark[0]{Aufgaben starten}{Aufgabenstart}
Nachdem die Aufgabe erstellt und  gespeichert ist kannst Du den Aufgabeneditor mit \bmg{Fliegen} verlassen.
Der Flug ist jetzt aktiv und auf der Karte erscheinen die gewählten Sektoren, Start- und Zielline gelb transparent hinterlegt.
Starten erfolgt, indem Du über die festgelegte Startlinie fliegst.

In dem Moment, wo \xc~detektiert, dass Du im Sektor bist, erscheint eine Meldung ''im Sektor''.
\xc~gibt Dir in den infoboxen und auf der Karte den optimalen Weg zum Abflugpunkt an.
Mit \bmt{Abflug}{bereit!} machst Du \xc~ nun scharf. Es werden die nächsten Wendepunkte eingebelendet
und alles auf den Start vorbereitet.

In dem Moment, wo Du jetzt die Startlinie überfliegst, erscheint die Meldung  ''Aufgabenstart''  zusammen mit Höhe,
Geschwindigkeit und Uhrzeit.

Willst Du pokern, kannst Du mit \bmt{Abflug}{verschieben} den Abflug wieder zurücksetzen.
Mit \bmg{Abflugpunkt} kannst Du zum letzten als Abflug markierten Punkt zurücknavigieren. Normalerweise nimmt \xc~den letzten überflug über die Startlinie als Start und wird von da ab rechnen.

\subsection*{\textcolor{blue}{AAT-Aufgaben - $\Delta$- Zeit  / Punkt verschieben}}\pdfbookmark[0]{AAT-Aufgaben}{aattarget}
\bc\nav{2}{Zielpunkt}{}\ec

Hier werden der Reihe nach die Zielpunkte samt Area in Vergrößerung dargestellt.
Ganz wichtig hierbei ist das kleine Kästchen ''optimiert'' am unteren linken Bildrand.
Wenn es angekreuzt ist, setzt Dir \xc~ den Kurs so, das entsprechend Deiner MC-Einstellung
und Deiner $\Delta$-Zeit, die Du unter den Voreinstellungen eingegeben hast,
am Ziel ankommen wirst. Wohlgemerkt, \xc~ ist kein Wetterfrosch und kann eine nahende Front
etc.\ nicht erkennen!

Um diese Punkte verschieben zu können mußt Du daher das ''Optimiert''--Kästchen eben
 {\textbf nicht} ankreuzen. Oben rechts siehst Du den notwendigen Schnitt, um pünktlich
 anzukommen, oben links die $Delta$--Zeit: positiv, wenn Ankunft OK, negativ Ankunft wenn zu früh.
 
 Die automatische Weiterschaltung der Wendepunkte erfolgt in dem Moment, wo Du in den Sektor geflogen bist,
%%%%%--------------------------------------------------------------------%%%%%
%%%%%--------------------------------------------------------------------%%%%%
%%%%%--------------------------------------------------------------------%%%%%
%%%%%--------------------------------------------------------------------%%%%%
%%%%%--------------------------------------------------------------------%%%%%
\newpage\section{Karte und Lufträume Anzeige/Deaktivieren}
%%%%%--------------------------------------------------------------------%%%%%
Mit den folgenden Klicks werden die Anzeigen auf der Karte ein oder ausgeschaltet.
\bc
\ansi{2}{Gelände}{Aus/Ein}\quad
\ansi{2}{Luftraum}{Aus/Ein}\quad
\ansi{2}{Topo}{Aus/Ein}
\ec
\subsection{\textcolor{blue}{Geländequerschnitt in Flugrichtung}}\pdfbookmark[0]{Gelaendequerschnitt}{Gelaendequerschnitt}
\xc~kann einen Geländequerschnitt (ähnlich eines Barogramm mit Geländehöhe) über den Task legen.
\bc\sk\blink\bmg{Aussehen}\blink\bmw{Pages}\ec

Unter \bmw{Bottom Area} kann dann z.B. für die oben ausgewählten InfoBoxSeiten \bmw{Cross Section} angewählt werden.
Neben der aktuellen Höhenlinie werden auch evtl. vorhandene Lufträume und die Entfernung angezeigt.
Ein nettes feature!
\subsection*{\textcolor{blue}{Lufträume vor Flugbeginn deaktivieren}}\index{Lufträume}\pdfbookmark[0]{Lufträume deaktivieren}{luftraum}

Hier gibt es zwei Möglichkeiten:

\subsubsection*{Global (über die Filterfunktion)}
\bc\konf{3}{Luftraum}{Einstellungen}\ec

Es erscheint das Luftraum-Fenster, auf dem eingeben kann,ob und welche Lufträume angezeigt werden sollen.
Dazu \bmw{Warnen}  oder \bmw{Anzeigen} etc. für den jeweiligen Luftraum anklicken.

Ein weiterer Klick auf \bmg{Suche} und man gelangt in das Suchfenster der Lufträume.

Mithilfe der Filterfunktion (z.B.\ ''Entfernung'' 250km) können nun sämtliche Lufträume für diesen Tag vor dem Start
deaktiviert werden, wenn man möchte. Das Wiederaktivieren erfolgt in umgekehrter Reihenfolge.


\subsubsection*{Lokal (direkt über den Bildschirm)}
Klicke {\textbf einmal} auf die Stelle der Karte, wo sich eine Luftraumgrenze befindet.
Es erscheint das Fenster ''Kartenelemente an diesem Ort''. Anschließend auf  \bmw{Details} und bei
Bedarf mit \bmg{Best. Tag} deaktivieren. Ab V6.6 kann sogar direkt  im Fenster deaktiviert werden.

Wieder aktivieren geht dann aber nur über \bmw{Details}.

%%%%%--------------------------------------------------------------------%%%%%
%%%%%--------------------------------------------------------------------%%%%%
%%%%%--------------------------------------------------------------------%%%%%
%%%%%--------------------------------------------------------------------%%%%%
%%%%%--------------------------------------------------------------------%%%%%
\section{Infobox im Detail - Was ist was}\pdfbookmark[0]{Infoboxen}{infoboxen}
%%%%%--------------------------------------------------------------------%%%%%
\subsection*{\textcolor{blue}{Infoboxen-Seiten anordnen (hochkant, quer, etc\dots)}}\pdfbookmark[0]{Infoboxseiten}{Infoboxseiten}
\bc\sk\blink\bmw{Aussehen}\blink\bmw{Anordnung}\ec
\subsection*{\textcolor{blue}{Infoboxen ändern/anpassen}}\pdfbookmark[0]{Infobox ändern}{InfoboxChange}
\bc\sk\blink\bmw{Aussehen}\blink\bmw{InfoBox Modi}\ec
Hier auf die entsprechende Seite klicken und Du kannst Deine InfoBoxen
den Feldern zuordenen.
%%%%%--------------------------------------------------------------------%%%%%
%%%%%--------------------------------------------------------------------%%%%%
%%%%%--------------------------------------------------------------------%%%%%
%%%%%--------------------------------------------------------------------%%%%%
%%%%%--------------------------------------------------------------------%%%%%
\newpage\section{Bildnachweis}
%%%%%--------------------------------------------------------------------%%%%%
\subsection*{\textcolor{blue}{Geländequerschnitt (Cross-Section) aktiviert}}
In diesem Bild seht Ihr \xc~Karte zentral, Cross-Section am unteren Bildschirm aktiviert. Nett, da auch gleich die Bezeichnungen 
der Lufträume dabei sind.
\begin{center}
 \includegraphics[angle=0,width=0.5\linewidth,keepaspectratio='true']{figures/CrossSection.png}
 % CrossSection.png: 489x645 pixel, 72dpi, 17.25x22.75 cm, bb=0 0 489 645
\end{center}
%%%%%--------------------------------------------------------------------%%%%%
%%%%%--------------------------------------------------------------------%%%%%
%%%%%--------------------------------------------------------------------%%%%%
%%%%%--------------------------------------------------------------------%%%%%
%%%%%--------------------------------------------------------------------%%%%%
\newpage\section{Generell}\label{ch:generell}\pdfbookmark[0]{Generelles-WICHTIG!}{generelles}
%%%%%--------------------------------------------------------------------%%%%%
\begin{itemize}
\item \xc~ hat pauschal zwei Modi: Aufgabe und ''herumgedödel''. (Intern gibt es Aufgabe, Kurbeln, Endanflug und Aufgabenabbruch)
Wenn eine Aufgabe aktiv bzw. geladen ist, beziehen sich viele Berechnungen grundsätzlich auf
das Umrunden der Aufgabe.
\item \xc~ rechnet ''um die Ecke'', d.h.\ die Endanflugshöhe auf dem Bildschirm links bezieht
sich immer auf die Vervollständigung der Aufgabe (also dem Umrunden aller Wendepunkte bis zum Ziel)--
nicht dem Erreichen des nächsten Wegpunktes.
\item Die Berechnungen während einer Aufgabe finden grundsätzlich mit dem gesetzten (automatisch oder manuell)
MC--Wert statt. Im Aufgabenabbruchmodus kann ein anderer ''Sicherheits' 'MC--Wert festgelegt
werden, auf den  \xc~ dann automatisch zurückführt.
\item Im Aufgabenabbruchmodus zeigt der Endanflugspfeil die benötigte Höhe des nächst\-ge\-le\-ge\-nen
(falls nicht explizit gewählt) Wegpunktes, andernfalls diejenige des z.B.\ unter
Alternativen ausgewählten Punktes an.
\item \textcolor{blue}{Es gibt mitunter Konfusion während des Endanfluges mit MC--Werten größer 0. \\
\xc~ halt sich streng an die reine MC--Lehre. Das bedeutet, daß ein Endanflug ein reines Gleiten ist, auf dem {\textsf nicht mehr gekurbelt} wird. Das ist gleichbedeutend mit MC=0.
Die ''übliche Vorgehensweise'' und dies speziell bei Gegenwind, den MC--Wert als Verschiebung der Polare zu mißbrauchen kann
daher zu außerordentlich verwirrenden Ergebnissen führen.}

\vspace{1em}
\xc~wird z.B.  bei einem 0.2m Bart und 21km/h Gegenwind errechnen, daß Du nie ans Ziel kommst, da der Versatz größer ist,
als die gewonnene Höhe. Aus diesem Grunde kann genau diese Funktion unter

\vspace{1em}
\bc\sk\blink\bmw{Endanflugrechner}\blink\bmw{Endanflugrechner}\blink\bmw{geschätzte Winddrift}\ec

\vspace{1em}
ausgeschaltet werden und mit dem ''üblichen'' Verhalten anderer Endanflugrechner gerechnet werden.

Es fehlt bislang eine ''was wäre wenn'' Funktion, die es erlaubt, auch beim Gleiten mit Gegenwind und
evtl.\ noch Kurbeln den Endanflug genügend genau kalkulieren zu können, wenn man die Vorflug--Geschwindigkeit variiert.
\item In den unzähligen Infoboxen können sowohl Informationen für eine aktivierte Aufgabe als auch für einen Wegpunkt außerhalb
einer Aufgabe eingegeben werden.
\end{itemize}
\printindex
\end{document}

