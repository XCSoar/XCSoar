This is step-by-step instructions on how to cross compile a Kobo kernel to modify the
kernel, add modules or both.  This is needed for OTG support.

Although most, if not all, Kobo hardware supports OTG, most models have OTG disabled
in the kernel.  Even the one model (as of Jan 2022) that has it enabled is missing many
modules.  By following this guide, OTG support can be added.  Note older Kobos use 
uImage but newer ones use zImage.

The following has been tested for only one model of Kobo (ClaraHD) and only one XCSoar
development environment (Debian 11 with sudo, git and the two ide/provisioning scripts
run).

\section{Install additional tools on development machine}
Login to the development machine and install tools as follows:

\begin{verbatim}
sudo apt-get install bc
sudo apt-get install lzop
\end{verbatim}

\section{Get kernel information}
On the relevant model of Kobo, make sure it is updated and loaded with XCSoar.  Boot
the Kobo and at the splash screen choose Network then connect to WiFi and enable Telnet.
Get the IP address from the Wifi menu then connect to the Kobo with telnet using PuTTY
or similar terminal program.  From there get the data as follows:

\begin{verbatim}
cat /proc/version
\end{verbatim}

The text returned contains both the kernel version and the gcc version.
For everything that follows, use only the kernel version triplet such as 4.1.15
and ignore anything after the ``-``.  For the gcc version use the triplet such as 5.3.0.

Next get the kernel config file as follows:

\begin{verbatim}
zcat /proc/config.gz >/mnt/onboard/config
\end{verbatim}

Download the /mnt/onboard/config file to your home directory.

\section{Install toolchain for arm-linux-gnueabihf}
To compile the kernel you must install a toolchain (gcc, g++ etc.) of the version with
which the kernel was compiled (as determined above).  There are many toolchains available
and you can either install a binary or compile one.  An example using linaro follows:

\begin{verbatim}
go to 
  https://releases.linaro.org/components/toolchain/binaries
select the most recently dated folder for the 
  compiler version determined above (e.g. 5.3-2016.05)
select the platform as arm-linux-gnueabihf
download the tar.xz file for x86_64_arm_linux_gnueabihf to /opt
  Note the third number of the version may differ
  (e.g. 5.3.1 instead of 5.3.0)
cd /opt
sudo tar -xvf <filename>
\end{verbatim}

Do not add the toolchain to the path but note the directory that contains the binaries such as
gcc and g++.  Also, make note of the prefix on the binaries (such as arm-linux-gnueabihf-).

\section{Install kernel source}
Slightly outdated but suitable sources for the kernels for the different models are kept
on GitHub in the Kobolabs/Kobo-Reader repository.  Note it is filed by chipset though the
model may also be shown.  If needed, to find the right chipset google for the Kobo wiki.

Download the appropriate .bz2 file to your home directory then enter the following commands:

\begin{verbatim}
cd
sudo tar -xvf <filename>
cd kernel
sudo make CROSS_COMPILE=<binary path>/<binary prefix> \
  ARCH=arm distclean
\end{verbatim}

Add the config file to the source and update it.

\begin{verbatim}
cd
cp config ./kernel/.config
cd kernel
sudo make CROSS_COMPILE=<binary path>/<binary prefix> \
  ARCH=arm oldconfig
\end{verbatim}

\section{Build the kernel}
It is now ready to attempt to build the original kernel.
If there are errors, edits must be made to
successfully compile the kernel.

\begin{verbatim}
sudo make CROSS_COMPILE=<binary path>/<binary prefix> \
  ARCH=arm clean
sudo make CROSS_COMPILE=<binary path>/<binary prefix> \
  ARCH=arm vmlinux
\end{verbatim}

If you get an error "/usr/bin/ld: scripts/dtc/dtc-parser.tab.o:(.bss+0x50):
multiple definition of `yylloc'" it is because older toolchains default
to warn on duplicate definitions and newer ones fail.  Correct it as follows:

\begin{verbatim}
sudo nano scripts/dtc/dtc-parser.tab.c
  change
    YYLTYPE yylloc;
  to
    extern YYLTYPE yylloc;
sudo make CROSS_COMPILE=<binary path>/<binary prefix> \
  ARCH=arm vmlinux
\end{verbatim}

If you get an error "<filename><address> error: implicit declaration of function"
for function pinctrl\_pm\_select\_sleep\_state or
pinctrl\_pm\_select\_default\_state or pinctrl\_pm\_lookup\_state it is because
older toolchains default to warn on implicit declarations and newer ones fail. 
Correct it as follows:

\begin{verbatim}
sudo nano <filename>
  add
    #include <linux/pinctrl/consumer.h>
  to the bottom of the #include <linux...> list
sudo make CROSS_COMPILE=<binary path>/<binary prefix> \
  ARCH=arm vmlinux
\end{verbatim}

If you get an error about missing imx6sll-e80k02-base.dts the e60 file is the
same as the e80 file for our purposes so use it as follows:

\begin{verbatim}
cd arch/arm/boot/dts
ln -s imx6sll-e60k02-base.dts ims6sll-e80k02-base.dts
sudo make CROSS_COMPILE=<binary path>/<binary prefix> \
  ARCH=arm vmlinux
sudo make CROSS_COMPILE=<binary path>/<binary prefix> \
  ARCH=arm modules
sudo make CROSS_COMPILE=<binary path>/<binary prefix> \
  ARCH=arm zImage
\end{verbatim}

\section{Edit the kernel configuration and rebuild}
At this point you should have a successful build of factory kernel.  Now it is necessary
to edit the kernel as needed to support the desired OTG capabilities.  The main
functions and drivers which should be supported are OTG, sound, network and  serial.
Specifically for Clara HD, OTG was already enabled so this involved adding modules, not
built in functionality.  The changes were made as follows:

\begin{verbatim}
sudo make CROSS_COMPILE=<binary path>/<binary prefix> \
  ARCH=arm menuconfig
Go to Device Drivers | USB support
Change USB Serial Converter support to "M"
Go to USB Serial Converter support
Change AIRcable Bluetooth... to "M"
Change Winchiphead CH341... to "M"
Change CP210x family... to "M"
Change FTDI Single... to "M"
Change Garmin GPS Driver to "M"
Change Moschip 7840... to "M"
Change Navman GPS device... to "M"
Change Prolific 2303 Single... to "M"
Back out to Device Drivers
Go to Sound card support
Go to Advanced Linux Sound Architecture
Go to USB sound devices
Change USB Audio/MIDI driver to "M"
Back out to Device Drivers
Go to Network device support
Change USB Network Adapters to "M"
Go to USB Network Adapters
Change Multi-purpose USB Networking... to "M"
Change ASIX AX88xxx... to " "
Change ASIX AX88179... to " "
Change CDC EEM support to "M"
Change CDC NCM support (NEW) to "M"
Change NetChip 1080... to " "
Change Host for RNDIS... to "M"
Change Simple USB Network... to "M"
Change eTEK based... to " "
Change Embedded ARM... to " "
Change Sharp Zaurus... to " "
save as .config (default) then exit
sudo make CROSS_COMPILE=<binary path>/<binary prefix> \
  ARCH=arm vmlinux
sudo make CROSS_COMPILE=<binary path>/<binary prefix> \
  ARCH=arm zImage
sudo make CROSS_COMPILE=<binary path>/<binary prefix> \
  ARCH=arm modules
\end{verbatim}

\section{Create the file for the XCSoar build}

If you are building a new kernel (i.e. original kernel doesn't support OTG), the desired
file is created already as the zImage file and must be copied or moved to the
/opt/kobo/kernel folder on the XCSoar build machine.  If the factory kernel already
supports OTG, the desired file is a file of loadable modules for the factory kernel.
To create the file of loadable modules, proceed as follows for the example of Clara HD:

\begin{verbatim}
cd ~/kernel
sudo install -d -m 775 -g root -o root \
  modules
sudo install -d -m 775 -g root -o root \
  modules/<kernel version>
sudo install -d -m 775 -g root -o root \
  modules/<kernel version>/drivers
sudo install -d -m 775 -g root -o root \
  modules/<kernel version>/drivers/usb
sudo install -d -m 775 -g root -o root \
  modules/<kernel version>/drivers/usb/serial
sudo install -D -m 644 -g root -o root \
  drivers/usb/serial/*.ko \
  modules/<kernel version>/drivers/usb/serial
sudo install -d -m 775 -g root -o root \
  modules/<kernel version>/drivers/net
sudo install -d -m 775 -g root -o root \
  modules/<kernel version>/drivers/net/usb
sudo install -D -m 644 -g root -o root \
  drivers/net/*.ko \
  modules/<kernel version>/drivers/net
sudo install -D -m 644 -g root -o root \
  drivers/net/usb/*.ko \
  modules/<kernel version>/drivers/net/usb
sudo install -d -m 775 -g root -o root \
  modules/<kernel version>/sound
sudo install -d -m 775 -g root -o root \
  modules/<kernel version>/sound/usb
sudo install -D -m 644 -g root -o root \
  sound/usb/*.ko \
  modules/<kernel version>/sound/usb
sudo install -d -m 775 -g root -o root \
  modules/<kernel version>/sound/core
sudo install -D -m 644 -g root -o root \
  sound/core/*.ko \
  modules/<kernel version>/sound/core
tar -czvf ClaraHD_OTG_Drivers.tgz modules
sudo mkdir /opt/kobo
sudo mkdir /opt/kobo/kernel
sudo cp ClaraHD_OTG_Drivers.tgz /opt/kobo/kernel
\end{verbatim}
