\chapter{Configuration}\label{cha:configuration}
XCSoar is a highly configurable glide computer and can be customised
to suit a wide variety of preferences and user requirements.  This
chapter describes the configuration settings and options.

\section{Scope of configuration}

There are several ways XCSoar can be customised:
\begin{itemize}

\item Modifying configuration settings.  This is the sort of configuration
 most likely to be performed by users; and this is given the greatest attention in this document.
\item Changing the language, or even just to change the wording
  of text in the user interface.
\item Changing the button assignments and button menus.  This allows 
the content and structure of the button menu to be changed. 
\item Changing or adding actions performed when glide computer events
 take place.
\item Defining how long status messages appear and sounds to be played
 when those messages occur.
\end{itemize}
Describing all of these is beyond the scope of this document;
the user is referred to the {\em XCSoar Advanced Configuration Manual} 
for more details. \todonum[inline]{Write a advanced config manual} 

\section{Modifying settings}

There are a large set of configuration settings that may be customised
from the Settings dialog accessible from the menu under
\begin{quote}
\bmenu{Config}\blink\bmenu{Config}\blink\bmenu{Setup system}
\end{quote}

You are strongly discouraged from changing these settings during
flight.  \warning  All changes to the settings should be performed on the ground
so that their desired effect on the programs behaviour can be
verified.

The settings dialog contains several pages.  Once changes have been made,
click the Close button on the screen or PWR/ESC on Altair to close the dialog
and return the program back to normal map mode.

\tip Once you are happy with your configuration settings, save the
profile file and make a backup so that you can later restore the
settings if your PDA's memory is accidentally erased.

See Chapter~\ref{cha:data-files} for a description of the data formats
of files referred to in the settings.  Where no file is to be used,
the field can be left blank.  File name fields in forms show files
that match a file extension filter.  This makes it much easier to find
and select the correct file.

The main configuration dialog (Setup System) can be run in Basic or
Expert user level, via a selectable field on the left of the dialog.
When in Basic mode, many of the less commonly used and advanced
configuration settings are hidden.  In the descriptions below,
all of the parameters marked with an asterix are only visible in
expert user level.

Basic user level:
\begin{center}
\includegraphics[angle=0,width=0.8\linewidth,keepaspectratio='true']{figures/config-basic.png}
\end{center}

Expert user level:
\begin{center}
\includegraphics[angle=0,width=0.8\linewidth,keepaspectratio='true']{figures/config-expert.png}
\end{center}

\subsection*{Safety lock}

A safety feature is available to prevent settings being modified in
flight.  This optionally prevents the configuration settings dialog
from starting if the aircraft is in flight. See Section~\ref{sec:interface} for customisation.

Note: In simulator mode the configuration settings dialog will also be available in flight.

%\subsection*{Fail-safe}

%If the XCSoar software crashes due to an unrecoverable error while
%loading a file, the file will be removed from the configuration
%settings in order to prevent the crash reoccuring.  Therefore, if an
%error was found in a file, the user must re-enter that file in the
%configuration settings after remedying the situation.

\clearpage
\section{Site}
The dialog specifies most of the important files that must be
configured when flying at a new site.

\begin{center}
\includegraphics[angle=0,width=0.8\linewidth,keepaspectratio='true']{figures/config-site.png}
\end{center}

\begin{description}
\item[Map Database]  The name of the map file (XCM) containing digital elevation
  terrain data, topology, and optionally waypoints, airspace etc. A good
  prepared database file covers all the needs for this page.
\item[Waypoints]  Primary waypoints file.  If left blank, waypoints are loaded
from the map file (if available).
\item[More Waypoints]  Secondary waypoints file.  This may be used to add waypoints for a competition.
\item[Airspaces]  The file name of the primary airspace file.  If left blank,
airspaces are loaded from the map file (if available).
\item[More Airspaces]  The file name of the secondary airspace file.
\item[Terrain file*]  The name of the file containing digital elevation
  terrain data.  Typically left blank, because terrain is loaded from the map
  file.
\item[Topology file*]  Specifies the file defining the topological features.
The topology file defines the map topology in terms of points, lines
and areas with optional labels.  Typically left blank, because topology is
loaded from the map file.
\item[Waypoint Details*]  The airfields file may contain extracts from Enroute Supplements or
other contributed information about individual airfields.
\end{description}

Airspace files define Special Use Airspace.  Up to two files may be
specified, the first for the main SUA file, and the second is intended
for use with NOTAM airspace, and is referred to as the additional
airspace file.

The XCM map database concept is the recommended way to setup a site to fly.
The old method (XCSoar v5.x) requires each to be separate files and to be
specified separately (as the ``Terrain file'' and ``Topology file'' respectively).  

When XCM map files are used, however, then these files contain terrain, topology
and optionally waypoints.  In this case, the ``Terrain file'', ``Topology file'' and 
``Primary waypoint file'' may be left blank and the system will load those items
from the map file. However, if a map file is used, the use can still specify the other
files and they will be used instead of the data in the map file.

See Section~\ref{sec:map} for more details on map files.


\clearpage
\section{Airspace}

This page is used to determine how the airspace information is
displayed and how warnings are issued.

\begin{center}
\includegraphics[angle=0,width=0.8\linewidth,keepaspectratio='true']{figures/config-airspace.png}
\end{center}

\begin{description}
\item[Airspace display] Controls how airspace display and warnings are filtered based on altitude.  The airspace filter dialog also allows filtering
 of display and warnings independently for each airspace class.
\begin{description}
\item[All on] All the airspace information is displayed at the same time.
\item[Clip] Only airspace below a user determined altitude is shown.
\item[Auto] Only airspace at the current altitude plus or minus a user definable margin is shown.
\item[All Below]  Only airspace below the glider is shown.
\end{description}
\item[Clip altitude] For clip mode, this is the altitude below which airspace is displayed.
\item[Margin] For auto mode, this is the safety margin for warnings and display.
\item[Warnings] Determines whether all warnings are enabled or disabled.
\item[Warning time*]  This is the time before an incursion is estimated at
  which the system will warn the pilot.
\item[Acknowledge time*]  This is the time period in which an acknowledged airspace warning will not be repeated.
\item[Use black outline*] Draws a black outline around each airspace
\end{description}

This page also has \button{Colours} and \button{Filter} buttons which
can be used to review or change the colours/patterns used by each
airspace class, and whether each airspace class will be filtered out
of warnings and/or display.  

\subsection*{Colours}
This function is used to determine the colours used to draw each type of
airspace.

First select the airspace type you wish to change.

\begin{center}
\includegraphics[angle=0,width=0.8\linewidth,keepaspectratio='true']{figures/config-airspacecolors.png}
\end{center}

Pressing the \button{Lookup} button brings up the airspace select dialog.
This functions similarly to the waypoint lookup dialog, and allows
search based on name, distance, direction, and type (class).  

\begin{center}
\includegraphics[angle=0,width=0.8\linewidth,keepaspectratio='true']{figures/airspacelookup.png}
\end{center}

Now select the colour and pattern you wish the selected airspace to be drawn in.


\subsection*{Filters}
The filter function is described in Section~\ref{sec:airsp-filt-dial}.

%\clearpage
%\section{Airspace Colours and Patterns}


%This is accessed via the configuration dialog, menu under
%\bmenu{Config}\blink\bmenu{Setup system} in the Airspace page, select the
%button \button{Colours}

%%%%%%%%%%%%%%%%%%
\clearpage
\section{Map Display}\label{sec:map-display}

This page provides options relating to the map display.

\begin{center}
\includegraphics[angle=0,width=0.8\linewidth,keepaspectratio='true']{figures/config-map.png}
\end{center}

\begin{description}
\item[Labels] This setting \label{conf:labels} determines the label
displayed with each waypoint. There are 7 options:

\begin{description}
\item[Names] The full name of each waypoint is displayed.
\item[Numbers] The waypoint enumeration of each waypoint is displayed, as read
from the waypoint file. Referring to this number is not reliable, thus the
feature will die soon.
\item[None] No names are displayed with the waypoints.
\item[Names in task] Names are only displayed for waypoints that are in the active task as well as the home airfield.
\item[First 3] The first 3 letters of the waypoint name are displayed.
\item[First 5] The first 5 letters of the waypoint name are displayed.
\item[First word] Only the first word (up to the first space) of the waypoint name is displayed.
\end{description}

\item[Trail length] \label{conf:snailtrail} Determines whether and how long a
snail trail is drawn behind the glider.
\begin{description}
\item[Off] No trail is drawn
\item[Long] A long trail is drawn (approx 60 minutes)
\item[Short] A short trail is drawn (approx 10 minutes) 
\item[Full] A trail for the entire flight is drawn
\end{description}

\item[Cruise/Circling Orientation] \label{conf:orientation} This determines how
the screen is rotated with the glider, depending on it's current display mode.
\begin{description}
\item[North up] The moving map display will always be orientated true north to
south and the glider icon will be rotated to show its course (corrected for
wind).
\item[Track up] The moving map display will be rotated so the glider's track
 is oriented up. The north arrow symbol points to true north. The glider symbol 
 may be shown rotated according to the computed heading of the glider taking wind into account.
\item[Target up] The moving map display will be rotated so the current target
direction is oriented up.
\end{description}

\item[Circling zoom] \label{conf:circlingzoom} This determines whether separate
zoom levels will be maintained for circling and cruise modes.  If enabled, then the 
map will zoom in automatically when entering circling mode and zoom out
automatically when leaving circling mode.

\item[Trail drift*] \label{conf:traildrift} Determines whether the
snail trail is drifted with the wind when displayed in circling mode.  When OFF,
the snail trail is uncompensated for wind draft.

\item[Trail type*] \label{conf:snailtype} Sets the type of the snail trail
display.
\begin{description}
\item[Vario \#1]  Within lift areas lines get displayed green and
thicker, while sinking lines are shown red and thin.  Zero lift
is presented as a grey line.
\item[Vario \#2]  The climb colour for this scheme is orange to red, sinking is
displayed as light blue to dark blue. Zero lift
is presented as a yellow line.
\item[Altitude] The colour scheme corresponds to the height.
\end{description}

\item[Trail width*] \label{conf:trailwidth} Sets the width of the snail trail
display.
\item[Detour cost markers*]  If enabled this displays in cruise flight some
figures projected in front of the nose of the glider icon.  This is the
additional distance in percent if you fly up the position of the figure and
after that again straight towards the target, compared to the straight distance
to target.
\item[FLARM map]  \label{conf:flarm-on-map}This enables the display of FLARM
traffic on the map window as well as the pop-up radar-like display.
\begin{description}
\item[OFF]  FLARM map display disabled
\item[ON/Fixed]  FLARM map display enabled with fixed scale.
\item[ON/Scaled]  FLARM map display enabled and auto scaled. The FLARM targets on the map display are 
scaled so that when the map is at large zoom levels, targets are still visible.
\end{description}
\end{description}


%%%%%%%%%%%%%%%%%%
\clearpage
\section{Symbols}\label{sec:symbols}

This page provides options relating to the items overlaying the map display.

\begin{center}
\includegraphics[angle=0,width=0.8\linewidth,keepaspectratio='true']{figures/config-symbols.png}
\end{center}

\begin{description}
\item[Glider position] \label{conf:gliderposition} Defines the location of the
glider drawn on the screen in percent from the bottom. 

\item[Final glide bar]  Two styles are available: Default and
Alternate. The differences between these styles is cosmetic.  Alternate displays the height difference to the 
right of the final glide bar; default displays the height difference above/below the final glide bar and inside a 
rounded box.
\item[Landable fields] \label{conf:waypointicons} Three styles are available:
Purple circles (WinPilot style), a high contrast (monochrome) style with icons,
or orange icons. See Section~\ref{sec:waypoint-schemes} for details

\item[Wind arrow*]  Determines the way the wind arrow is drawn on the map.
\begin{description}
\item[Arrow head] Draws an arrow head only
\item[Full arrow] Draws an arrow head with a dashed arrow line
\end{description}
\item[North arrow*]  Two styles are available.  Normal, or with a white outline.
\end{description}

%%%%%%%%%%%%%%%%%%
\clearpage
\section{Terrain display}\label{sec:terrain-display}

This page sets how terrain and topology is drawn on the map window.

\begin{center}
\includegraphics[angle=0,width=0.8\linewidth,keepaspectratio='true']{figures/config-terrain.png}
\end{center}

\begin{description}
\item[Terrain display]  Draws digital elevation terrain on the map.
\item[Topology display]  Draws topological features (roads, rivers, lakes) on
the map.
\item[Slope shading]  \label{conf:shading} Slopes faced to the wind get
displayed brighter and the averted slopes get darker.
\item[Terrain contrast]  Defines the amount of phong shading in the terrain rendering.  Use large values 
to emphasise terrain slope, smaller values if flying in steep mountains.
\item[Terrain brightness]  Defines the brightness (whiteness) of the terrain rendering.  This controls the 
average illumination of the terrain.
\item[Terrain colors]  Defines the colour ramp used in terrain rendering.  Various schemes are available, 
which works best for you will depend on how mountainous your region is.
\end{description}


\begin{maxipage}
The available terrain colour schemes are illustrated in the table below.

\begin{longtable}{c c c c}
\includegraphics[angle=0,width=3.5cm,keepaspectratio='true']{figures/ramp-terrain-flatlands.png}&
\includegraphics[angle=0,width=3.5cm,keepaspectratio='true']{figures/ramp-terrain-mountanous.png}&
\includegraphics[angle=0,width=3.5cm,keepaspectratio='true']{figures/ramp-terrain-icao.png}&
\includegraphics[angle=0,width=3.5cm,keepaspectratio='true']{figures/ramp-terrain-grey.png}
\\

\includegraphics[angle=0,width=3.5cm,keepaspectratio='true']{figures/ramp-terrain-imhof4.png}&
\includegraphics[angle=0,width=3.5cm,keepaspectratio='true']{figures/ramp-terrain-imhof7.png}&
\includegraphics[angle=0,width=3.5cm,keepaspectratio='true']{figures/ramp-terrain-imhof12.png}&
\includegraphics[angle=0,width=3.5cm,keepaspectratio='true']{figures/ramp-terrain-imhofatlas.png}
\\
\end{longtable}
\end{maxipage}


%%%%%%%%%%%%%%%%%%
\clearpage
\section{Glide computer}\label{sec:final-glide}

This page allows glide computer algorithms to be configured.

\begin{center}
\includegraphics[angle=0,width=0.8\linewidth,keepaspectratio='true']{figures/config-glidecomputer.png}
\end{center}

\begin{description}
\item[Auto wind]  \label{conf:autowind} This allows switching on or off the
automatic wind algorithm.
 \begin{description}
\item[Manual]  When the algorithm is switched off, the pilot is responsible for
  setting the wind estimate.
\item[Circling]  Circling mode requires only a GPS source, 
\item[ZigZag]  ZigZag requires an intelligent vario with airspeed output.
\item[Both]  Uses Circling and ZigZag.
\end{description}

\item[External wind]  If enabled, the wind vector received from external
devices overrides XCSoar's internal wind calculation.

\item[Auto MC mode] This option defines which auto MacCready algorithm is used.
For more details see Section~\ref{sec:auto-maccready}.
 \begin{description}
\item[Final glide] Final glide adjusts MC for fastest arrival.
\item[Trending Average climb]  Sets MC to the trending average climb rate
based on all climbs.
\item[Both] Uses trending average during task, then fastest arrival when in
final glide mode.
\end{description}

\item[Block speed to fly*] If enabled, the command speed in cruise
  is set to the MacCready speed to fly in no vertical air-mass movement.
  If disabled, the command speed in cruise is set to the dolphin speed to fly,
  equivalent to the MacCready speed with vertical air-mass movement.

\item[Nav by baro altitude*] When enabled and if connected to a barometric
  altimeter, barometric altitude is used for all navigation functions. Otherwise
  GPS altitude is used.

\item[Flap forces cruise*]
  When this option is enabled, causes the flap switches in Vega to
  force cruise mode when the flap is not positive.  This means that
  when departing a thermal, switching to neutral or negative flap will
  immediately switch XCSoar's mode to cruise mode.

  Similarly, for Borgelt B50 systems, the speed command switch forces
  XCSoar's climb or cruise mode.

\item[L/D Average period] Average glide ratio is always calculated in real time. Here you can decide 
on how many seconds of flight this calculation must be done.
\end{description}

\section{Safety factors}

This page allows the safety heights and behaviour in abort mode to be defined.

\begin{center}
\includegraphics[angle=0,width=0.8\linewidth,keepaspectratio='true']{figures/config-safety.png}
\end{center}

\begin{description}
\item[Arrival height]  The height above terrain that the glider
  should arrive at for a safe landing.
\item[Terrain height] \label{conf:safetyterrain} The height above terrain that the glider must
  clear during final glide.
\item[Safety MC*]  The MacCready setting used for reach calculations, task abort, alternates and
  for determining arrival altitude at airfields. 
\item[STF risk factor*] 
  The STF risk factor reduces the MacCready setting used to calculate
  speed to fly as the glider gets low, in order to compensate for
  risk.  Set to 0.0 for no compensation, 1.0 scales MC linearly with
  height.  See Section~\ref{sec:speed-fly-with} for more details.
\end{description}
See Section~\ref{sec:safety-heights} for more details on the meanings
of the safety heights.

\section{Route}

This page allows control over glide reach calculations and route
optimisations.

\begin{description}
\item[Route mode] \label{conf:routemode} This controls which types
of obstacles are used in route planning.
\item[Route climb] \label{conf:routeclimb} When enabled and Mc is positive, route planning allows climbs between the
  aircraft location and destination.
\item[Route ceiling] \label{conf:routeceiling} When enabled, route planning climbs are limited to ceiling defined by greater of current aircraft altitude plus 500 m and the thermal ceiling.  If disabled, climbs are unlimited.
\item[Reach mode] \label{conf:turningreach} How calculations are performed of the reach of the glider with respect to terrain. {\bf Off} Reach calculations disabled. {\bf Straight} The reach is from straight line paths from the glider. {\bf Turning} The reach is calculated allowing turns around terrain obstacles.
\item[Reach display]  \label{conf:gliderange} This determines whether the
glide reach is drawn as a line resp.\ a shade on the map area.
\item[Reach polar] \label{conf:reachpolar} This determines the glide performance used in reach, landable arrival, abort and alternate calculations: {\bf Task} Uses task glide polar; {\bf Safety MC} Uses safety MacCready value.
\end{description}

%%%%%%%%%%%%%%%%%%
\clearpage
\section{Polar}

This page allows the glide polar to be defined.

\begin{center}
\includegraphics[angle=0,width=0.8\linewidth,keepaspectratio='true']{figures/config-polar.png}
\end{center}

\begin{description}
\item[Type]  \label{conf:polar} This contains a selection of gliders of
different performance classes, as well as a special entry for ``External Polar File''.  
\item[Polar file]  When ``External Polar File'' is the polar type, 
 this is the name of the file containing the glide polar data in WinPilot
 format.
\item[V rough air] The maximum manoeuvring speed can be entered on this page to prevent the glide computer from commanding unrealistic cruise speeds.
\item[Handicap] The handicap factor used for OnLine Contest scoring.
\item[Dump time] The time in seconds needed for dumping full ballast.
\end{description}

\clearpage
\section{Devices} \label{conf:comdevices}

The Devices page is used to specify the ports used to communicate with
the GPS and other serial devices. The default settings are COM1 and
4800 bits per second.  When connected to the Vega intelligent
variometer, the settings should be COM1 and 38400.

\begin{center}
\includegraphics[angle=0,width=0.8\linewidth,keepaspectratio='true']{figures/config-devices.png}
\end{center}

Two COM devices are available (device A and device B), to allow, for
example, one to be connected to a GPS and another to be connected to a
second device such as a variometer.  If there is no second device, set
the device B port settings to the same as those of device A -- this
instructs the program to ignore device B.

The specific type of device can also be selected from a list in order
to enable support for devices with proprietary protocols or special
functions.

COM ports 0 to 10 may be used.  Which COM port is appropriate for you
depends on what make of PDA you use, and the communications medium
(serial cable, BlueTooth, virtual COM port, SD card or CF based GPS,
internal GPS).  Detailing the various options for different devices is
beyond the scope of this document.  If you have trouble identifying
which COM port to set, please refer to the XCSoar website and mailing
lists.

\begin{description}
\item[Use GPS time] This option, if enabled sets the clock of the computer to the GPS time once a fix is set.  
This is only necessary if your computer does not have a real-time clock with battery backup or your computer 
frequently runs out of battery power or otherwise loses time.
\item[Ignore checksum*] If your GPS device outputs invalid NMEA checksums, this will allow it's data to be used anyway.
\end{description}

%%%%%%%%%%%%%%%%%%
\clearpage
\section{Units}

This page allows you to set the units preferences used in all
displays, {\InfoBox}es, dialogs and input fields.  Separate selections
are available for speed, distance, lift rate, altitude, temperature, task
speed and latitude/longitude.

\begin{center}
\includegraphics[angle=0,width=0.8\linewidth,keepaspectratio='true']{figures/config-units.png}
\end{center}

The UTC offset field allows the UTC local time offset to be specified.
The local time is displayed below in order to make it easier to verify
the correct offset has been entered.  Offsets to the half-hour may be
set.


%%%%%%%%%%%%%%%%%%%%%%%%%

\clearpage
\section{Interface}\label{sec:interface}

This page allows to customize the way the user controls and interacts with
XCSoar.

\begin{center}
\includegraphics[angle=0,width=0.8\linewidth,keepaspectratio='true']{figures/config-interface.png}
\end{center}

\begin{description}
\item[Auto Blank] This determines whether to blank the display after a long
period of inactivity when operating on internal battery power (visible for PDA
only).

\item[Safety lock]  This determines whether the configuration settings dialog is
accessible during flight.
\item[Gestures]  Enable this, if you run XCSoar on a touch-screen device and
like the gestures to control it.
\item[Events*]  The Input Events file defines the menu system and how XCSoar
responds to button presses and events from external devices.
\item[Language*]  The language file defines translations for XCSoar text in English to
other languages.  Select ``None'' for a native English interface, ``Automatic''
to localize XCSoar according to the system settings.
\item[Status message*]  The status message file can be used to define sounds to be played when certain
events occur, and how long various status messages will appear on screen.
\item[Menu timeout*]  This determines how long menus will appear on screen if the user
does not make any button presses or interacts with the computer.
\item[Debounce timeout*]  This is the minimum interval between the system recognising key presses. 
Set this to a low value for a more responsive user interface; if
it is too low, then accidental multiple key presses can occur.

\item[Text Input Style*] Determines which style for text entries is used. See Section~ for further information on text entries.
\begin{description}
\item[HighScore Style] For entering text you have to change the underlined character to the relevant letter.
\item[Keyboard] Uses the on-screen keyboard for entering text.
\item[Default] Uses the default input style for your platform.
\end{description}
\end{description}

Some Pocket PC devices have poorly designed keys that are subject to
accidental multiple key presses, which is known as key `bouncing'.  The
de-bounce timeout sets a minimum time between successive key presses
that is detected by XCSoar, to alleviate this problem.  If this value
is set very high, then the user interface will feel unresponsive; if
the value is set too low, then bouncing may occur.

Press the \button{Fonts} button to adjust the fonts XCSoar uses.

%%%%%%%%%%

\clearpage
\section{Fonts}

This page enables customisation of fonts in various fields of the programm.

\begin{center}
\includegraphics[angle=0,width=0.8\linewidth,keepaspectratio='true']{figures/config-fonts.png}
\end{center}

Once the customisation is enabled, the \button{Edit} buttons allow to change some parameters (Font 
Face, Height, Bold and Italic) of the chosen font.

If customisation is disabled, default fonts will be used.

%%%%%%%%%%
\clearpage
\section{Layout}

This page defines various display styles used by symbols and InfoBoxes.

\begin{center}
\includegraphics[angle=0,width=0.8\linewidth,keepaspectratio='true']{figures/config-layout.png}
\end{center}

\begin{description}
\item[Infobox Geometry]  Sets the geometry values for infoboxes. In landscpae
mode infoboxes are placed left and right, in portrait mode top and bottom of the screen. The numbers in front refer 
to the total number of infoboxes.
\item[Msg window*]  Defines the alignment of the status message box, either
centered or in the top left corner.
\item[Dialog Style*]  Determines the display size of dialogs.
\end{description}

%%%%%%%%%%

\clearpage
\section{FLARM and other gauges}\label{sec:vario-gauge}

\begin{center}
\includegraphics[angle=0,width=0.8\linewidth,keepaspectratio='true']{figures/config-othergauges.png}
\end{center}
\label{conf:variogauge}
\begin{description}
\item[Speed arrows]  Whether to show speed command arrows on the Vario gauge.
When shown, in cruise mode, arrows point up to command slow down; arrows point down to command speed up.
\item[Show average]  Whether to show the average climb rate.  In cruise mode, this switches to showing the average 
netto airmass rate.
\item[Show MacCready]  Whether to show the MacCready setting.
\item[Show bugs*]  Whether to show the bugs percentage.
\item[Show ballast*]  Whether to show the ballast percentage.
\item[Show gross*]  Whether to show the gross vario value.
\item[Averager needle*]  If true, the vario gauge will display a hollow averager
needle.  During cruise, this needle displays the average netto value.  During circling, this needle displays the average gross value.
\item[FLARM radar]  \label{conf:flarmdisplay} Enables the display of the FLARM
radar gauge. The track bearing of the target relative to the track bearing of the aircraft 
is displayed as an arrow head.
\item[Auto close FLARM]  This will close the FLARM radar view when all
FLARM traffic has gone.
\item[ThermalAssistant] \label{conf:thermalassistant} Enables the display of the
ThermalAssistant gauge.
\end{description}

In all FLARM environment, the colour of the target indicates the threat level.

%%%%%%%%%%

\clearpage
\section{Default Task Rules}
Task rules may be defined to limit valid starts according to competition
rules. \label{conf:taskrules}

\begin{center}
\includegraphics[angle=0,width=0.8\linewidth,keepaspectratio='true']{figures/config-rules.png}
\end{center}

\begin{description}
\item[Start max speed]  Maximum speed allowed in start observation zone.  Set to 0 for no limit.
\item[Start max speed margin] Maximum speed above maximum start speed to tolerate.  Set to 0 for no tolerance.
\item[Start max height]  Maximum height above ground while starting the task.  Set to 0 for no limit.
\item[Start max height margin]  Maximum height above maximum start height to tolerate.  Set to 0 for no tolerance.
\item[Start height ref]  Reference used for start max height rule
\begin{description}
\item[MSL] Reference is altitude above mean sea level
\item[AGL] Reference is the height above the start point
\end{description}
\item[Finish min height]  Minimum height above ground while finishing the task.  Set to 0 for no limit. 
\item[Online contest] Determines the rules used to optimise On-Line Contest
paths.  The implementation  conforms to the official release 2010, Sept. 23.
\begin{description}
\item[OLC Sprint]  Up to 5 points including start and finish, maximum duration
2.5 hours, finish height must not be below start height.
\item[OLC FAI]  Four points with common start and finish.  For tasks longer than
500km, no leg less than 25\% or larger than 45\%; otherwise no leg less than 28\% of total.  Finish height must 
not be lower than start height less 1000 meters.
\item[OLC Classic]  Up to seven points including start and finish, finish height
must not be lower than start height less 1000 meters.
\item[OLC League]  A contest on top of the classic task optimization, cutting
a 2.5 hours segment over max. 3 of the turns. Finish height must not be below
start height.
\item[OLC Plus]  A combination of Classic and FAI rules. 30\% of the FAI score
are added to the Classic score.
\end{description}
\end{description}

%%%%%%%%%%

\clearpage
\section{{\InfoBox}es}

This page allows the configuration of four InfoBoxe sets to be defined for each
display mode (circling, cruise, final glide) and one auxiliary set.  See
Section~\ref{cha:infobox} for a description of the infobox types and their meanings.

\begin{center}
\includegraphics[angle=0,width=0.8\linewidth,keepaspectratio='true']{figures/config-infoboxes.png}
\end{center}

To arrange a set of InfoBoxes press one of the buttons labeled with the name
of the set.  The InfoBoxes are numbered; the location of the InfoBoxes depends
on the screen geometry.  The table below shows the infobox numbers for landscape screen layout (Altair):

\begin{tabular}{|c|c|}
\hline
1 &  \\
\hline
2 &  \\
\hline
3 &  \\
\hline
4 & 7 \\
\hline
5 & 8 \\
\hline
6 & 9 \\
\hline
\end{tabular}

The table below shows the infobox numbering for portrait screen layout:

\begin{tabular}{|c|c|c|c|}
\hline
1 & 2 & 3 & 4 \\
\hline
\hline
5 & 6 & 7 & 8 \\
\hline
\end{tabular}

\begin{description}
\item[Inverse InfoBoxes*]  If true, the InfoBoxes are white on black, otherwise black on white.
\item[Colour InfoBoxes*]  If true, certain InfoBoxes will have coloured text. For example, the 
active waypoint infobox will be blue when the glider is above final glide.
\item[Infobox border*]  Two styles for infobox borders are available: `Box'
draws boxes around each infobox.  `Tab' draws a tab at the top of the infobox across the title.
\end{description}

%%%%%%%%%%

\clearpage
\section{Logger}

This page allows you to set the pilot and aircraft details used for
annotating XCSoar's IGC logger. 
% The fields available are the pilot's name, aircraft type, and aircraft registration.

\begin{center}
\includegraphics[angle=0,width=0.8\linewidth,keepaspectratio='true']{figures/config-logger.png}
\end{center}

\begin{description}
\item[Time step cruise*]  This is the time interval between logged points when not circling. 
\item[Time step circling*]  This is the time interval between logged points when circling. 
\item[Pilot name]  This is the pilot name used in the internal software logger declaration.
\item[Aircraft type]  This is the aircraft type used in the internal software logger declaration.
%\item[Aircraft rego]  This is the aircraft registration used in the internal software logger declaration.
\item[Competition ID]  This is the aircraft competition ID.
\item[Logger ID]  This is the logger registration.
\item[Short file name]  This determines whether the logger uses the short or the long IGC file name.
\item[Auto logger*]  Enables the automatic starting and stopping of the logger
on takeoff and landing respectively. Disable when flying paragliders to prevent the low ground speeds from
triggering the automatic logger.
\end{description}

%%%%%%%%%%

\clearpage
\section{Experimental features}

This page provides experimental features which are not finished yet.

\begin{center}
\includegraphics[angle=0,width=0.8\linewidth,keepaspectratio='true']{figures/config-exp.png}
\end{center}

\begin{description}
\item[Device model] This setting allows the adaptation to specific hardware
XCSoar runs on (visible for PDA only).
\end{description}
